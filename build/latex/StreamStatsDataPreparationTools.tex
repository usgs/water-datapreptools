%% Generated by Sphinx.
\def\sphinxdocclass{report}
\documentclass[letterpaper,10pt,english]{sphinxmanual}
\ifdefined\pdfpxdimen
   \let\sphinxpxdimen\pdfpxdimen\else\newdimen\sphinxpxdimen
\fi \sphinxpxdimen=.75bp\relax

\PassOptionsToPackage{warn}{textcomp}
\usepackage[utf8]{inputenc}
\ifdefined\DeclareUnicodeCharacter
% support both utf8 and utf8x syntaxes
\edef\sphinxdqmaybe{\ifdefined\DeclareUnicodeCharacterAsOptional\string"\fi}
  \DeclareUnicodeCharacter{\sphinxdqmaybe00A0}{\nobreakspace}
  \DeclareUnicodeCharacter{\sphinxdqmaybe2500}{\sphinxunichar{2500}}
  \DeclareUnicodeCharacter{\sphinxdqmaybe2502}{\sphinxunichar{2502}}
  \DeclareUnicodeCharacter{\sphinxdqmaybe2514}{\sphinxunichar{2514}}
  \DeclareUnicodeCharacter{\sphinxdqmaybe251C}{\sphinxunichar{251C}}
  \DeclareUnicodeCharacter{\sphinxdqmaybe2572}{\textbackslash}
\fi
\usepackage{cmap}
\usepackage[T1]{fontenc}
\usepackage{amsmath,amssymb,amstext}
\usepackage{babel}
\usepackage{times}
\usepackage[Bjarne]{fncychap}
\usepackage{sphinx}

\fvset{fontsize=\small}
\usepackage{geometry}

% Include hyperref last.
\usepackage{hyperref}
% Fix anchor placement for figures with captions.
\usepackage{hypcap}% it must be loaded after hyperref.
% Set up styles of URL: it should be placed after hyperref.
\urlstyle{same}
\addto\captionsenglish{\renewcommand{\contentsname}{Contents:}}

\addto\captionsenglish{\renewcommand{\figurename}{Fig.\@ }}
\makeatletter
\def\fnum@figure{\figurename\thefigure{}}
\makeatother
\addto\captionsenglish{\renewcommand{\tablename}{Table }}
\makeatletter
\def\fnum@table{\tablename\thetable{}}
\makeatother
\addto\captionsenglish{\renewcommand{\literalblockname}{Listing}}

\addto\captionsenglish{\renewcommand{\literalblockcontinuedname}{continued from previous page}}
\addto\captionsenglish{\renewcommand{\literalblockcontinuesname}{continues on next page}}
\addto\captionsenglish{\renewcommand{\sphinxnonalphabeticalgroupname}{Non-alphabetical}}
\addto\captionsenglish{\renewcommand{\sphinxsymbolsname}{Symbols}}
\addto\captionsenglish{\renewcommand{\sphinxnumbersname}{Numbers}}

\addto\extrasenglish{\def\pageautorefname{page}}

\setcounter{tocdepth}{1}



\title{StreamStats Data Preparation Tools}
\date{Nov 08, 2019}
\release{4.0beta}
\author{}
\newcommand{\sphinxlogo}{\vbox{}}
\renewcommand{\releasename}{Release}
\makeindex
\begin{document}

\pagestyle{empty}
\sphinxmaketitle
\pagestyle{plain}
\sphinxtableofcontents
\pagestyle{normal}
\phantomsection\label{\detokenize{index::doc}}



\chapter{Data Preparation Workflow}
\label{\detokenize{workflow:data-preparation-workflow}}\label{\detokenize{workflow::doc}}

\section{First Steps}
\label{\detokenize{firstSteps:first-steps}}\label{\detokenize{firstSteps::doc}}
\noindent\sphinxincludegraphics{{doggo}.jpg}


\chapter{StreamStats\_DataPrep ESRI Toolbox}
\label{\detokenize{StreamStats_DataPrep:streamstats-dataprep-esri-toolbox}}\label{\detokenize{StreamStats_DataPrep::doc}}

\section{Database Setup}
\label{\detokenize{StreamStats_DataPrep:database-setup}}\index{databaseSetup (class in StreamStats\_DataPrep)@\spxentry{databaseSetup}\spxextra{class in StreamStats\_DataPrep}}

\begin{fulllineitems}
\phantomsection\label{\detokenize{StreamStats_DataPrep:StreamStats_DataPrep.databaseSetup}}\pysigline{\sphinxbfcode{\sphinxupquote{class }}\sphinxcode{\sphinxupquote{StreamStats\_DataPrep.}}\sphinxbfcode{\sphinxupquote{databaseSetup}}}
Set up the workspace needed to process elevation and hydrography data.

This tool is a wrapper on {\hyperref[\detokenize{databaseSetup:databaseSetup.databaseSetup}]{\sphinxcrossref{\sphinxcode{\sphinxupquote{databaseSetup.databaseSetup()}}}}}.
\subsubsection*{Methods}


\begin{savenotes}\sphinxatlongtablestart\begin{longtable}{\X{1}{2}\X{1}{2}}
\hline

\endfirsthead

\multicolumn{2}{c}%
{\makebox[0pt]{\sphinxtablecontinued{\tablename\ \thetable{} -- continued from previous page}}}\\
\hline

\endhead

\hline
\multicolumn{2}{r}{\makebox[0pt][r]{\sphinxtablecontinued{Continued on next page}}}\\
\endfoot

\endlastfoot

{\hyperref[\detokenize{StreamStats_DataPrep:StreamStats_DataPrep.databaseSetup.getParameterInfo}]{\sphinxcrossref{\sphinxcode{\sphinxupquote{getParameterInfo}}}}}(self)
&
Database Setup inputs.
\\
\hline
\end{longtable}\sphinxatlongtableend\end{savenotes}
\index{getParameterInfo() (StreamStats\_DataPrep.databaseSetup method)@\spxentry{getParameterInfo()}\spxextra{StreamStats\_DataPrep.databaseSetup method}}

\begin{fulllineitems}
\phantomsection\label{\detokenize{StreamStats_DataPrep:StreamStats_DataPrep.databaseSetup.getParameterInfo}}\pysiglinewithargsret{\sphinxbfcode{\sphinxupquote{getParameterInfo}}}{\emph{self}}{}
Database Setup inputs.
\begin{quote}\begin{description}
\item[{Parameters}] \leavevmode\begin{description}
\item[{\sphinxstylestrong{Output Workspace}}] \leavevmode{[}DEWorkspace (File System){]}
Folder-type workspace for local folders and geodatabase to be created.

\item[{\sphinxstylestrong{Main ArcHydro Geodatabase Name}}] \leavevmode{[}GPString{]}
Name of the geodatabase to be created in “Output Workspace.”

\item[{\sphinxstylestrong{Hydrologic Unit Boundary Dataset}}] \leavevmode{[}DEShapefile or DEFeatureClass{]}
Polygon vector defining local processing units. Should have columns for outwalls and inwalls, see below.

\item[{\sphinxstylestrong{Outwall Field}}] \leavevmode{[}Field{]}
Field in “Hydrologic Unit Boundary Dataset” used to determine local folders and outwalls.

\item[{\sphinxstylestrong{Inwall Field}}] \leavevmode{[}Field{]}
Field in “Hydrologic Unit Boundary Dataset” used to determine inwalls.

\item[{\sphinxstylestrong{Hydrologic Unit Buffer Distance (m)}}] \leavevmode{[}GPString{]}
Distance to buffer local folder polygons by.

\item[{\sphinxstylestrong{Input Hydrography Workspace}}] \leavevmode{[}DEWorkspace{]}
Path to folder type workspace with geodatabases of NHD hydrography.

\item[{\sphinxstylestrong{Elevation Dataset Template}}] \leavevmode{[}DERasterBand{]}
Raster dataset to pull projection information from, works best as an ESRI grid.

\item[{\sphinxstylestrong{Alternative Outwall Buffer}}] \leavevmode{[}GPString (optional){]}
Distance for alternative outwall buffer.

\end{description}

\item[{Returns}] \leavevmode\begin{description}
\item[{\sphinxstylestrong{parameters}}] \leavevmode{[}list{]}
List of input parameters passed to the execute method.

\end{description}

\end{description}\end{quote}

\end{fulllineitems}


\end{fulllineitems}



\section{Make Elevation Data Index}
\label{\detokenize{StreamStats_DataPrep:make-elevation-data-index}}\index{makeELEVDATAIndex (class in StreamStats\_DataPrep)@\spxentry{makeELEVDATAIndex}\spxextra{class in StreamStats\_DataPrep}}

\begin{fulllineitems}
\phantomsection\label{\detokenize{StreamStats_DataPrep:StreamStats_DataPrep.makeELEVDATAIndex}}\pysigline{\sphinxbfcode{\sphinxupquote{class }}\sphinxcode{\sphinxupquote{StreamStats\_DataPrep.}}\sphinxbfcode{\sphinxupquote{makeELEVDATAIndex}}}
Create a seamless raster mosaic dataset from input digital elevation tiles.

This tool is a wrapper on {\hyperref[\detokenize{elevationTools:elevationTools.elevIndex}]{\sphinxcrossref{\sphinxcode{\sphinxupquote{elevationTools.elevIndex()}}}}}.
\subsubsection*{Methods}


\begin{savenotes}\sphinxatlongtablestart\begin{longtable}{\X{1}{2}\X{1}{2}}
\hline

\endfirsthead

\multicolumn{2}{c}%
{\makebox[0pt]{\sphinxtablecontinued{\tablename\ \thetable{} -- continued from previous page}}}\\
\hline

\endhead

\hline
\multicolumn{2}{r}{\makebox[0pt][r]{\sphinxtablecontinued{Continued on next page}}}\\
\endfoot

\endlastfoot

{\hyperref[\detokenize{StreamStats_DataPrep:StreamStats_DataPrep.makeELEVDATAIndex.getParameterInfo}]{\sphinxcrossref{\sphinxcode{\sphinxupquote{getParameterInfo}}}}}(self)
&
Make ELEV data index inputs
\\
\hline
\end{longtable}\sphinxatlongtableend\end{savenotes}
\index{getParameterInfo() (StreamStats\_DataPrep.makeELEVDATAIndex method)@\spxentry{getParameterInfo()}\spxextra{StreamStats\_DataPrep.makeELEVDATAIndex method}}

\begin{fulllineitems}
\phantomsection\label{\detokenize{StreamStats_DataPrep:StreamStats_DataPrep.makeELEVDATAIndex.getParameterInfo}}\pysiglinewithargsret{\sphinxbfcode{\sphinxupquote{getParameterInfo}}}{\emph{self}}{}
Make ELEV data index inputs
\begin{quote}\begin{description}
\item[{Parameters}] \leavevmode\begin{description}
\item[{\sphinxstylestrong{Output Geodatabase}}] \leavevmode{[}DEWorkspace (Geodatabase){]}
Path to the geodatabase that will hold the output raster mosaic dataset.

\item[{\sphinxstylestrong{Output Raster Mosaic Dataset Name}}] \leavevmode{[}GPString{]}
Name of raster mosaic dataset to output, defaults to IndexRMD.

\item[{\sphinxstylestrong{Coordinate System}}] \leavevmode{[}GPCoordinateSystem{]}
Coordinate system of input grids and raster mosaic dataset.

\item[{\sphinxstylestrong{Input Elevation Data workspace}}] \leavevmode{[}DEWorkspace (Folder){]}
Path to folder holding input digital elevation models to be included in the raster mosaic dataset.

\end{description}

\item[{Returns}] \leavevmode\begin{description}
\item[{\sphinxstylestrong{parameters}}] \leavevmode{[}list{]}
List of input parameters passed to the execute method.

\end{description}

\end{description}\end{quote}

\end{fulllineitems}


\end{fulllineitems}



\section{Extract Polygons}
\label{\detokenize{StreamStats_DataPrep:extract-polygons}}\index{ExtractPoly (class in StreamStats\_DataPrep)@\spxentry{ExtractPoly}\spxextra{class in StreamStats\_DataPrep}}

\begin{fulllineitems}
\phantomsection\label{\detokenize{StreamStats_DataPrep:StreamStats_DataPrep.ExtractPoly}}\pysigline{\sphinxbfcode{\sphinxupquote{class }}\sphinxcode{\sphinxupquote{StreamStats\_DataPrep.}}\sphinxbfcode{\sphinxupquote{ExtractPoly}}}
Extract a hydrologic unit from a digital elevation model based on a clipping polygon.

This tool is a wrapper on {\hyperref[\detokenize{elevationTools:elevationTools.extractPoly}]{\sphinxcrossref{\sphinxcode{\sphinxupquote{elevationTools.extractPoly()}}}}}.
\subsubsection*{Methods}


\begin{savenotes}\sphinxatlongtablestart\begin{longtable}{\X{1}{2}\X{1}{2}}
\hline

\endfirsthead

\multicolumn{2}{c}%
{\makebox[0pt]{\sphinxtablecontinued{\tablename\ \thetable{} -- continued from previous page}}}\\
\hline

\endhead

\hline
\multicolumn{2}{r}{\makebox[0pt][r]{\sphinxtablecontinued{Continued on next page}}}\\
\endfoot

\endlastfoot

{\hyperref[\detokenize{StreamStats_DataPrep:StreamStats_DataPrep.ExtractPoly.getParameterInfo}]{\sphinxcrossref{\sphinxcode{\sphinxupquote{getParameterInfo}}}}}(self)
&
Extract Polygon inputs.
\\
\hline
\end{longtable}\sphinxatlongtableend\end{savenotes}
\index{getParameterInfo() (StreamStats\_DataPrep.ExtractPoly method)@\spxentry{getParameterInfo()}\spxextra{StreamStats\_DataPrep.ExtractPoly method}}

\begin{fulllineitems}
\phantomsection\label{\detokenize{StreamStats_DataPrep:StreamStats_DataPrep.ExtractPoly.getParameterInfo}}\pysiglinewithargsret{\sphinxbfcode{\sphinxupquote{getParameterInfo}}}{\emph{self}}{}
Extract Polygon inputs.
\begin{quote}\begin{description}
\item[{Parameters}] \leavevmode\begin{description}
\item[{\sphinxstylestrong{Output Workspace}}] \leavevmode{[}DEWorkspace (Folder){]}
Path to folder to work in.

\item[{\sphinxstylestrong{ELEVDATA Raster Mosaic Dataset}}] \leavevmode{[}DEMosaicDataset{]}
Path to the raster mosaic dataset holding the elevation data.

\item[{\sphinxstylestrong{Clip Polygon}}] \leavevmode{[}GPFeatureLayer{]}
Feature class of the watershed boundary being used for clipping.

\item[{\sphinxstylestrong{Output Grid}}] \leavevmode{[}GPString{]}
Name of the output ESRI grid, defaults to dem\_dd.

\end{description}

\item[{Returns}] \leavevmode\begin{description}
\item[{\sphinxstylestrong{parameters}}] \leavevmode{[}list{]}
List of input parameters passed to the execute method.

\end{description}

\end{description}\end{quote}

\end{fulllineitems}


\end{fulllineitems}



\section{Check For NoData Cells}
\label{\detokenize{StreamStats_DataPrep:check-for-nodata-cells}}\index{CheckNoData (class in StreamStats\_DataPrep)@\spxentry{CheckNoData}\spxextra{class in StreamStats\_DataPrep}}

\begin{fulllineitems}
\phantomsection\label{\detokenize{StreamStats_DataPrep:StreamStats_DataPrep.CheckNoData}}\pysigline{\sphinxbfcode{\sphinxupquote{class }}\sphinxcode{\sphinxupquote{StreamStats\_DataPrep.}}\sphinxbfcode{\sphinxupquote{CheckNoData}}}
Check for no data cells in a digital elevation model.

This tool is a wrapper on {\hyperref[\detokenize{elevationTools:elevationTools.checkNoData}]{\sphinxcrossref{\sphinxcode{\sphinxupquote{elevationTools.checkNoData()}}}}}.
\subsubsection*{Methods}


\begin{savenotes}\sphinxatlongtablestart\begin{longtable}{\X{1}{2}\X{1}{2}}
\hline

\endfirsthead

\multicolumn{2}{c}%
{\makebox[0pt]{\sphinxtablecontinued{\tablename\ \thetable{} -- continued from previous page}}}\\
\hline

\endhead

\hline
\multicolumn{2}{r}{\makebox[0pt][r]{\sphinxtablecontinued{Continued on next page}}}\\
\endfoot

\endlastfoot

{\hyperref[\detokenize{StreamStats_DataPrep:StreamStats_DataPrep.CheckNoData.getParameterInfo}]{\sphinxcrossref{\sphinxcode{\sphinxupquote{getParameterInfo}}}}}(self)
&
Check for no data inputs.
\\
\hline
\end{longtable}\sphinxatlongtableend\end{savenotes}
\index{getParameterInfo() (StreamStats\_DataPrep.CheckNoData method)@\spxentry{getParameterInfo()}\spxextra{StreamStats\_DataPrep.CheckNoData method}}

\begin{fulllineitems}
\phantomsection\label{\detokenize{StreamStats_DataPrep:StreamStats_DataPrep.CheckNoData.getParameterInfo}}\pysiglinewithargsret{\sphinxbfcode{\sphinxupquote{getParameterInfo}}}{\emph{self}}{}
Check for no data inputs.
\begin{quote}\begin{description}
\item[{Parameters}] \leavevmode\begin{description}
\item[{\sphinxstylestrong{InputGrid}}] \leavevmode{[}DERasterBand{]}
Path to raster dataset to examine.

\item[{\sphinxstylestrong{Workspace}}] \leavevmode{[}DEWorkspace (Geodatabase){]}
Geodatabase-type workspace to work in.

\item[{\sphinxstylestrong{Output Feature Layer}}] \leavevmode{[}GPString{]}
Name of output feature class, defaults to DEM\_NoDataSinks.

\end{description}

\item[{Returns}] \leavevmode\begin{description}
\item[{\sphinxstylestrong{parameters}}] \leavevmode{[}list{]}
List of input parameters passed to the execute method.

\end{description}

\end{description}\end{quote}

\end{fulllineitems}


\end{fulllineitems}



\section{Fill NoData Cells}
\label{\detokenize{StreamStats_DataPrep:fill-nodata-cells}}\index{FillNoData (class in StreamStats\_DataPrep)@\spxentry{FillNoData}\spxextra{class in StreamStats\_DataPrep}}

\begin{fulllineitems}
\phantomsection\label{\detokenize{StreamStats_DataPrep:StreamStats_DataPrep.FillNoData}}\pysigline{\sphinxbfcode{\sphinxupquote{class }}\sphinxcode{\sphinxupquote{StreamStats\_DataPrep.}}\sphinxbfcode{\sphinxupquote{FillNoData}}}
Fill no data cells in a digital elevation model.

This tool is a wrapper on {\hyperref[\detokenize{elevationTools:elevationTools.fillNoData}]{\sphinxcrossref{\sphinxcode{\sphinxupquote{elevationTools.fillNoData()}}}}}.
\subsubsection*{Notes}

This tool can be run iteratively to fully fill no data areas that are larger than one cell.
\subsubsection*{Methods}


\begin{savenotes}\sphinxatlongtablestart\begin{longtable}{\X{1}{2}\X{1}{2}}
\hline

\endfirsthead

\multicolumn{2}{c}%
{\makebox[0pt]{\sphinxtablecontinued{\tablename\ \thetable{} -- continued from previous page}}}\\
\hline

\endhead

\hline
\multicolumn{2}{r}{\makebox[0pt][r]{\sphinxtablecontinued{Continued on next page}}}\\
\endfoot

\endlastfoot

{\hyperref[\detokenize{StreamStats_DataPrep:StreamStats_DataPrep.FillNoData.getParameterInfo}]{\sphinxcrossref{\sphinxcode{\sphinxupquote{getParameterInfo}}}}}(self)
&
Fill no data inputs.
\\
\hline
\end{longtable}\sphinxatlongtableend\end{savenotes}
\index{getParameterInfo() (StreamStats\_DataPrep.FillNoData method)@\spxentry{getParameterInfo()}\spxextra{StreamStats\_DataPrep.FillNoData method}}

\begin{fulllineitems}
\phantomsection\label{\detokenize{StreamStats_DataPrep:StreamStats_DataPrep.FillNoData.getParameterInfo}}\pysiglinewithargsret{\sphinxbfcode{\sphinxupquote{getParameterInfo}}}{\emph{self}}{}
Fill no data inputs.
\begin{quote}\begin{description}
\item[{Parameters}] \leavevmode\begin{description}
\item[{\sphinxstylestrong{Workspace}}] \leavevmode{[}DEWorkspace (Folder){]}
Path to folder to work in.

\item[{\sphinxstylestrong{Input Grid}}] \leavevmode{[}DERasterBand{]}
Path to raster dataset with no data values to be filled, defaults to DEM\_NoDataSinks.

\item[{\sphinxstylestrong{Output Grid}}] \leavevmode{[}GPString{]}
Path to write out filled raster dataset to, defaults to DEM\_filled.

\end{description}

\item[{Returns}] \leavevmode\begin{description}
\item[{\sphinxstylestrong{parameters}}] \leavevmode{[}list{]}
List of input parameters passed to the execute method.

\end{description}

\end{description}\end{quote}

\end{fulllineitems}


\end{fulllineitems}



\section{Project and Scale}
\label{\detokenize{StreamStats_DataPrep:project-and-scale}}\index{ProjScale (class in StreamStats\_DataPrep)@\spxentry{ProjScale}\spxextra{class in StreamStats\_DataPrep}}

\begin{fulllineitems}
\phantomsection\label{\detokenize{StreamStats_DataPrep:StreamStats_DataPrep.ProjScale}}\pysigline{\sphinxbfcode{\sphinxupquote{class }}\sphinxcode{\sphinxupquote{StreamStats\_DataPrep.}}\sphinxbfcode{\sphinxupquote{ProjScale}}}
Project and scale a digital elevation model.

This tool is a wrapper on {\hyperref[\detokenize{elevationTools:elevationTools.projScale}]{\sphinxcrossref{\sphinxcode{\sphinxupquote{elevationTools.projScale()}}}}}.
\subsubsection*{Notes}

After scaling, this tool attempts to set the correct z-units; however, if you vertical units are different from your horizontal units it is advized to check the z-units manually.
\subsubsection*{Methods}


\begin{savenotes}\sphinxatlongtablestart\begin{longtable}{\X{1}{2}\X{1}{2}}
\hline

\endfirsthead

\multicolumn{2}{c}%
{\makebox[0pt]{\sphinxtablecontinued{\tablename\ \thetable{} -- continued from previous page}}}\\
\hline

\endhead

\hline
\multicolumn{2}{r}{\makebox[0pt][r]{\sphinxtablecontinued{Continued on next page}}}\\
\endfoot

\endlastfoot

{\hyperref[\detokenize{StreamStats_DataPrep:StreamStats_DataPrep.ProjScale.getParameterInfo}]{\sphinxcrossref{\sphinxcode{\sphinxupquote{getParameterInfo}}}}}(self)
&
Project and scale digital elevation model inputs.
\\
\hline
\end{longtable}\sphinxatlongtableend\end{savenotes}
\index{getParameterInfo() (StreamStats\_DataPrep.ProjScale method)@\spxentry{getParameterInfo()}\spxextra{StreamStats\_DataPrep.ProjScale method}}

\begin{fulllineitems}
\phantomsection\label{\detokenize{StreamStats_DataPrep:StreamStats_DataPrep.ProjScale.getParameterInfo}}\pysiglinewithargsret{\sphinxbfcode{\sphinxupquote{getParameterInfo}}}{\emph{self}}{}
Project and scale digital elevation model inputs.
\begin{quote}\begin{description}
\item[{Parameters}] \leavevmode\begin{description}
\item[{\sphinxstylestrong{InWorkSpace}}] \leavevmode{[}DEWorkspace (Folder){]}
Path to the folder to work in.

\item[{\sphinxstylestrong{InGrid}}] \leavevmode{[}DERasterBand{]}
Path to the raster dataset to project and scale.

\item[{\sphinxstylestrong{OutGrid}}] \leavevmode{[}GPString{]}
Name for the projected and scaled raster, defaults to dem\_raw.

\item[{\sphinxstylestrong{OutCoordSys}}] \leavevmode{[}GPCoordinateSystem{]}
Coordinate system to project the input raster to.

\item[{\sphinxstylestrong{OutCellSize}}] \leavevmode{[}analysis\_cell\_size{]}
Output cell size to project the input raster to, defaults to 10 horizontal map units.

\item[{\sphinxstylestrong{RegPt}}] \leavevmode{[}GPString{]}
Registration point for the projected raster, defaults to “0 0”.

\item[{\sphinxstylestrong{scaleFact}}] \leavevmode{[}GPString{]}
Scale factor to use to convert the projected raster to integers, defaults to 100. Consider using a larger scale factor as cell-size decreases.

\end{description}

\item[{Returns}] \leavevmode\begin{description}
\item[{\sphinxstylestrong{parameters}}] \leavevmode{[}list{]}
List of input parameters passed to the execute method.

\end{description}

\end{description}\end{quote}

\end{fulllineitems}


\end{fulllineitems}



\section{TopoGrid (optional)}
\label{\detokenize{StreamStats_DataPrep:topogrid-optional}}\index{TopoGrid (class in StreamStats\_DataPrep)@\spxentry{TopoGrid}\spxextra{class in StreamStats\_DataPrep}}

\begin{fulllineitems}
\phantomsection\label{\detokenize{StreamStats_DataPrep:StreamStats_DataPrep.TopoGrid}}\pysigline{\sphinxbfcode{\sphinxupquote{class }}\sphinxcode{\sphinxupquote{StreamStats\_DataPrep.}}\sphinxbfcode{\sphinxupquote{TopoGrid}}}
Condition an input DEM using a flowline dendrite prior to hydro-enforcement.

This tool is a wrapper on {\hyperref[\detokenize{topo_grid:topo_grid.topogrid}]{\sphinxcrossref{\sphinxcode{\sphinxupquote{topo\_grid.topogrid()}}}}}.
\subsubsection*{Notes}

This function turns the input DEM into a 3D point cloud, thinned using the VIP algorithm so that not all points are retained from the original DEM. The point cloud is used in conjunction with the supplied flowlines to re-interpolate a DEM that is aware of the location of the flowlines and their flow direction.

This is a computationally intensive function. Running it via ArcPro or Python 3 will be faster than using ArcMap or Python 2.
\subsubsection*{Methods}


\begin{savenotes}\sphinxatlongtablestart\begin{longtable}{\X{1}{2}\X{1}{2}}
\hline

\endfirsthead

\multicolumn{2}{c}%
{\makebox[0pt]{\sphinxtablecontinued{\tablename\ \thetable{} -- continued from previous page}}}\\
\hline

\endhead

\hline
\multicolumn{2}{r}{\makebox[0pt][r]{\sphinxtablecontinued{Continued on next page}}}\\
\endfoot

\endlastfoot

{\hyperref[\detokenize{StreamStats_DataPrep:StreamStats_DataPrep.TopoGrid.getParameterInfo}]{\sphinxcrossref{\sphinxcode{\sphinxupquote{getParameterInfo}}}}}(self)
&
TopoGrid inputs.
\\
\hline
\end{longtable}\sphinxatlongtableend\end{savenotes}
\index{getParameterInfo() (StreamStats\_DataPrep.TopoGrid method)@\spxentry{getParameterInfo()}\spxextra{StreamStats\_DataPrep.TopoGrid method}}

\begin{fulllineitems}
\phantomsection\label{\detokenize{StreamStats_DataPrep:StreamStats_DataPrep.TopoGrid.getParameterInfo}}\pysiglinewithargsret{\sphinxbfcode{\sphinxupquote{getParameterInfo}}}{\emph{self}}{}
TopoGrid inputs.
\begin{quote}\begin{description}
\item[{Parameters}] \leavevmode\begin{description}
\item[{\sphinxstylestrong{Output Workspace}}] \leavevmode{[}DEWorkspace (Geodatabase){]}
Path to a geodatabase to work in.

\item[{\sphinxstylestrong{Dissolved HUC8 boundary}}] \leavevmode{[}DEFeatureClass or DEShapefile{]}
Feature class to use in bounding the topogrid conditioning process.

\item[{\sphinxstylestrong{Topogrid Buffer Distance}}] \leavevmode{[}GPDouble{]}
Distance to buffer the input HUC8 boundary, in the units of the HUC8 boundary.

\item[{\sphinxstylestrong{12 Digit Hydrologic Unit Datasets if dissolved HUC8 boundary failed}}] \leavevmode{[}DEFeatureClass or DEShapefile (Optional){]}
List of HUC12 boundaries to split up TopoGrid computations if the whole domain fails.

\item[{\sphinxstylestrong{Dendritic Flowline Features}}] \leavevmode{[}DEFeatureClass or DEShapefile{]}
Dendrite used for enforcing flow direction in topogrid.

\item[{\sphinxstylestrong{Buffered and Projected Elevation Data}}] \leavevmode{[}DERasterBand or DERasterDataset{]}
Input digital elevation model to be conditioned using topogrid.

\item[{\sphinxstylestrong{Output Cell Size}}] \leavevmode{[}GPString{]}
Cell size for output digital elevation model, defualts to 10 horizontal map units.

\item[{\sphinxstylestrong{VIP Percentage}}] \leavevmode{[}GPString{]}
Thinning value used in the Very Important Points algorithm to decide how many points from the original raster are retained, defaults to 5 percent.

\item[{\sphinxstylestrong{SnapGrid}}] \leavevmode{[}DERasterBand (Optional){]}
Raster to snap output grid to.

\end{description}

\item[{Returns}] \leavevmode\begin{description}
\item[{\sphinxstylestrong{parameters}}] \leavevmode{[}list{]}
List of input parameters passed to the execute method.

\end{description}

\end{description}\end{quote}

\end{fulllineitems}


\end{fulllineitems}



\section{Bathymetic Gradient Setup}
\label{\detokenize{StreamStats_DataPrep:bathymetic-gradient-setup}}\index{SetupBathyGrad (class in StreamStats\_DataPrep)@\spxentry{SetupBathyGrad}\spxextra{class in StreamStats\_DataPrep}}

\begin{fulllineitems}
\phantomsection\label{\detokenize{StreamStats_DataPrep:StreamStats_DataPrep.SetupBathyGrad}}\pysigline{\sphinxbfcode{\sphinxupquote{class }}\sphinxcode{\sphinxupquote{StreamStats\_DataPrep.}}\sphinxbfcode{\sphinxupquote{SetupBathyGrad}}}
Prepare bathymetric gradient inputs for use in hydro-enforcement.

This tool is a wrapper on {\hyperref[\detokenize{make_hydrodem:make_hydrodem.bathymetricGradient}]{\sphinxcrossref{\sphinxcode{\sphinxupquote{make\_hydrodem.bathymetricGradient()}}}}}.
\subsubsection*{Notes}

The bathymetric gradient refers to generating a sloping area around the flowline dendrite that ensures the lanscape around the dendrite flows to the stream. This also adds a sloping surface to double-line streams and waterbodies to help insure proper drainage after hydro-enforcement.
\subsubsection*{Methods}


\begin{savenotes}\sphinxatlongtablestart\begin{longtable}{\X{1}{2}\X{1}{2}}
\hline

\endfirsthead

\multicolumn{2}{c}%
{\makebox[0pt]{\sphinxtablecontinued{\tablename\ \thetable{} -- continued from previous page}}}\\
\hline

\endhead

\hline
\multicolumn{2}{r}{\makebox[0pt][r]{\sphinxtablecontinued{Continued on next page}}}\\
\endfoot

\endlastfoot

{\hyperref[\detokenize{StreamStats_DataPrep:StreamStats_DataPrep.SetupBathyGrad.getParameterInfo}]{\sphinxcrossref{\sphinxcode{\sphinxupquote{getParameterInfo}}}}}(self)
&
Setup Bathymetric Gradient inputs.
\\
\hline
\end{longtable}\sphinxatlongtableend\end{savenotes}
\index{getParameterInfo() (StreamStats\_DataPrep.SetupBathyGrad method)@\spxentry{getParameterInfo()}\spxextra{StreamStats\_DataPrep.SetupBathyGrad method}}

\begin{fulllineitems}
\phantomsection\label{\detokenize{StreamStats_DataPrep:StreamStats_DataPrep.SetupBathyGrad.getParameterInfo}}\pysiglinewithargsret{\sphinxbfcode{\sphinxupquote{getParameterInfo}}}{\emph{self}}{}
Setup Bathymetric Gradient inputs.
\begin{quote}\begin{description}
\item[{Parameters}] \leavevmode\begin{description}
\item[{\sphinxstylestrong{Output Workspace}}] \leavevmode{[}DEWorkspace (Geodatabase){]}
Path to a geodatabase to work in.

\item[{\sphinxstylestrong{Digital Elevation Model (used for snapping)}}] \leavevmode{[}DERasterBand{]}
Path to a digital elevation model to use for aligning output grids to the rest of the project.

\item[{\sphinxstylestrong{Dissolved HUC8 Dataset}}] \leavevmode{[}DEFeatureClass{]}
Feature class of the local folder boundary.

\item[{\sphinxstylestrong{NHD Area}}] \leavevmode{[}DEFeatureClass{]}
Feature class of NHD double line streams.

\item[{\sphinxstylestrong{NHD Dendrite}}] \leavevmode{[}DEFeatureClass{]}
Feature class of the flowline dendrite.

\item[{\sphinxstylestrong{NHD Waterbody}}] \leavevmode{[}DEFeatureClass{]}
Feature class of the NHD water bodies.

\item[{\sphinxstylestrong{Cell Size}}] \leavevmode{[}GPString{]}
Output grid cell size, defaults to 10 horizontal map units.

\end{description}

\item[{Returns}] \leavevmode\begin{description}
\item[{\sphinxstylestrong{parameters}}] \leavevmode{[}list{]}
List of input parameters passed to the execute method.

\end{description}

\end{description}\end{quote}

\end{fulllineitems}


\end{fulllineitems}



\section{Coastal DEM Processing (optional)}
\label{\detokenize{StreamStats_DataPrep:coastal-dem-processing-optional}}\index{CoastalDEM (class in StreamStats\_DataPrep)@\spxentry{CoastalDEM}\spxextra{class in StreamStats\_DataPrep}}

\begin{fulllineitems}
\phantomsection\label{\detokenize{StreamStats_DataPrep:StreamStats_DataPrep.CoastalDEM}}\pysigline{\sphinxbfcode{\sphinxupquote{class }}\sphinxcode{\sphinxupquote{StreamStats\_DataPrep.}}\sphinxbfcode{\sphinxupquote{CoastalDEM}}}
Prepare coastal areas for hydro-enforcement.

This tool is a wrapper on {\hyperref[\detokenize{make_hydrodem:make_hydrodem.coastaldem}]{\sphinxcrossref{\sphinxcode{\sphinxupquote{make\_hydrodem.coastaldem()}}}}}.
\subsubsection*{Methods}


\begin{savenotes}\sphinxatlongtablestart\begin{longtable}{\X{1}{2}\X{1}{2}}
\hline

\endfirsthead

\multicolumn{2}{c}%
{\makebox[0pt]{\sphinxtablecontinued{\tablename\ \thetable{} -- continued from previous page}}}\\
\hline

\endhead

\hline
\multicolumn{2}{r}{\makebox[0pt][r]{\sphinxtablecontinued{Continued on next page}}}\\
\endfoot

\endlastfoot

{\hyperref[\detokenize{StreamStats_DataPrep:StreamStats_DataPrep.CoastalDEM.getParameterInfo}]{\sphinxcrossref{\sphinxcode{\sphinxupquote{getParameterInfo}}}}}(self)
&
Coastal digital elevation model processing inputs.
\\
\hline
\end{longtable}\sphinxatlongtableend\end{savenotes}
\index{getParameterInfo() (StreamStats\_DataPrep.CoastalDEM method)@\spxentry{getParameterInfo()}\spxextra{StreamStats\_DataPrep.CoastalDEM method}}

\begin{fulllineitems}
\phantomsection\label{\detokenize{StreamStats_DataPrep:StreamStats_DataPrep.CoastalDEM.getParameterInfo}}\pysiglinewithargsret{\sphinxbfcode{\sphinxupquote{getParameterInfo}}}{\emph{self}}{}
Coastal digital elevation model processing inputs.
\begin{quote}\begin{description}
\item[{Parameters}] \leavevmode\begin{description}
\item[{\sphinxstylestrong{Workspace}}] \leavevmode{[}DEWorkspace (Folder){]}
Path to folder to work in.

\item[{\sphinxstylestrong{Input raw DEM}}] \leavevmode{[}DERasterBand{]}
Original digital elevation model to be corrected for coastal areas, defaults to dem\_raw.

\item[{\sphinxstylestrong{Input LandSea polygon feature class}}] \leavevmode{[}DEFeatureClass{]}
Feature class indicating areas of land and sea.

\item[{\sphinxstylestrong{Output DEM}}] \leavevmode{[}DERasterBand{]}
Output digital elevation model name, defaults to dem\_sea.

\item[{\sphinxstylestrong{Sea Level}}] \leavevmode{[}GPString{]}
Value to insert into areas identified as the sea, defaults to -60000 vertical map units.

\end{description}

\item[{Returns}] \leavevmode\begin{description}
\item[{\sphinxstylestrong{parameters}}] \leavevmode{[}list{]}
List of input parameters passed to the execute method.

\end{description}

\end{description}\end{quote}

\end{fulllineitems}


\end{fulllineitems}



\section{Hydro-Enforce DEM}
\label{\detokenize{StreamStats_DataPrep:hydro-enforce-dem}}\index{HydroDEM (class in StreamStats\_DataPrep)@\spxentry{HydroDEM}\spxextra{class in StreamStats\_DataPrep}}

\begin{fulllineitems}
\phantomsection\label{\detokenize{StreamStats_DataPrep:StreamStats_DataPrep.HydroDEM}}\pysigline{\sphinxbfcode{\sphinxupquote{class }}\sphinxcode{\sphinxupquote{StreamStats\_DataPrep.}}\sphinxbfcode{\sphinxupquote{HydroDEM}}}
Hydro-Enforce a DEM.

This tool is a wrapper on {\hyperref[\detokenize{make_hydrodem:make_hydrodem.hydrodem}]{\sphinxcrossref{\sphinxcode{\sphinxupquote{make\_hydrodem.hydrodem()}}}}}.
\subsubsection*{Notes}

We suggest that AGREE defaults not be changed as this can lead to alignment issues between the flowlines and the resultant hydro-enforced DEM and subsequent products (FDR and FAC).
\subsubsection*{Methods}


\begin{savenotes}\sphinxatlongtablestart\begin{longtable}{\X{1}{2}\X{1}{2}}
\hline

\endfirsthead

\multicolumn{2}{c}%
{\makebox[0pt]{\sphinxtablecontinued{\tablename\ \thetable{} -- continued from previous page}}}\\
\hline

\endhead

\hline
\multicolumn{2}{r}{\makebox[0pt][r]{\sphinxtablecontinued{Continued on next page}}}\\
\endfoot

\endlastfoot

{\hyperref[\detokenize{StreamStats_DataPrep:StreamStats_DataPrep.HydroDEM.getParameterInfo}]{\sphinxcrossref{\sphinxcode{\sphinxupquote{getParameterInfo}}}}}(self)
&
\begin{quote}\begin{description}
\item[{Parameters}] \leavevmode
\end{description}\end{quote}

\\
\hline
\end{longtable}\sphinxatlongtableend\end{savenotes}
\index{getParameterInfo() (StreamStats\_DataPrep.HydroDEM method)@\spxentry{getParameterInfo()}\spxextra{StreamStats\_DataPrep.HydroDEM method}}

\begin{fulllineitems}
\phantomsection\label{\detokenize{StreamStats_DataPrep:StreamStats_DataPrep.HydroDEM.getParameterInfo}}\pysiglinewithargsret{\sphinxbfcode{\sphinxupquote{getParameterInfo}}}{\emph{self}}{}~\begin{quote}\begin{description}
\item[{Parameters}] \leavevmode\begin{description}
\item[{\sphinxstylestrong{Output Workspace}}] \leavevmode{[}DEWorkspace (geodatabase){]}
Geodatabase-type workspace where output raster will be saved.

\item[{\sphinxstylestrong{Scratch Workspace}}] \leavevmode{[}DEWorkspace (folder){]}
Folder-type scratch workspace.

\item[{\sphinxstylestrong{HUC layer}}] \leavevmode{[}DEFeatureClass{]}
Polygon feature class the bounds the local folder you are working in.

\item[{\sphinxstylestrong{Digital Elevation Model}}] \leavevmode{[}DERasterBand{]}
Digital elevation model to be hydro-enforced.

\item[{\sphinxstylestrong{Stream Dendrite}}] \leavevmode{[}DEFeatureClass{]}
Polyline feature class describing where streams are on the landscape.

\item[{\sphinxstylestrong{Snap Grid}}] \leavevmode{[}DERasterBand{]}
Raster grid used to align output DEM with other related grids or adjacent local folders.

\item[{\sphinxstylestrong{NHD Waterbody Grid}}] \leavevmode{[}DERasterDataset (optional){]}
Grid representing waterbodies from the bathymetric gradient step.

\item[{\sphinxstylestrong{NHD Flowline Grid}}] \leavevmode{[}DERasterDataset (optional){]}
Grid representing flowlines from the bathymetric gradient step.

\item[{\sphinxstylestrong{Inner Walls}}] \leavevmode{[}DEFeatureClass (optional){]}
Polyline feature class used to inforce internal drainage.

\item[{\sphinxstylestrong{Cell Size}}] \leavevmode{[}GPString{]}
Output cell size, defaults to 10.

\item[{\sphinxstylestrong{Drain Plugs}}] \leavevmode{[}DEFeatureClass (optional){]}
\item[{\sphinxstylestrong{HUC buffer}}] \leavevmode{[}GPDouble (optional){]}
Distance to buffer the HUC layer, dfaults to 50.

\item[{\sphinxstylestrong{Inner Wall Buffer}}] \leavevmode{[}GPDouble (optional){]}
Distance to buffer the inwall, defaults to 15.

\item[{\sphinxstylestrong{Inner Wall Height}}] \leavevmode{[}GPDouble (optional){]}
Inwall height, defaults to 150000.

\item[{\sphinxstylestrong{Outer Wall Height}}] \leavevmode{[}GPDouble (optional){]}
Outer wall height, defaults to 300000.

\item[{\sphinxstylestrong{AGREE buffer}}] \leavevmode{[}GPDouble (optional){]}
Defaults to 60.

\item[{\sphinxstylestrong{AGREE Smooth Drop}}] \leavevmode{[}GPDouble (optional){]}
Defaults to -500.

\item[{\sphinxstylestrong{AGREE Sharp Drop}}] \leavevmode{[}GPDouble (optional){]}
Defaults to -50000.

\item[{\sphinxstylestrong{Bowl Depth}}] \leavevmode{[}GPDouble (optional){]}
Defaults to 2000.

\end{description}

\item[{Returns}] \leavevmode\begin{description}
\item[{\sphinxstylestrong{parameters}}] \leavevmode{[}list{]}
List of input parameters passed to the execute method.

\end{description}

\end{description}\end{quote}

\end{fulllineitems}


\end{fulllineitems}



\section{Adjust Accumulation}
\label{\detokenize{StreamStats_DataPrep:adjust-accumulation}}\index{AdjustAccum (class in StreamStats\_DataPrep)@\spxentry{AdjustAccum}\spxextra{class in StreamStats\_DataPrep}}

\begin{fulllineitems}
\phantomsection\label{\detokenize{StreamStats_DataPrep:StreamStats_DataPrep.AdjustAccum}}\pysigline{\sphinxbfcode{\sphinxupquote{class }}\sphinxcode{\sphinxupquote{StreamStats\_DataPrep.}}\sphinxbfcode{\sphinxupquote{AdjustAccum}}}
Adjust flow accumulation grids following hydro-enforcement.

This tool is a wrapper on {\hyperref[\detokenize{make_hydrodem:make_hydrodem.adjust_accum}]{\sphinxcrossref{\sphinxcode{\sphinxupquote{make\_hydrodem.adjust\_accum()}}}}}.
\subsubsection*{Methods}


\begin{savenotes}\sphinxatlongtablestart\begin{longtable}{\X{1}{2}\X{1}{2}}
\hline

\endfirsthead

\multicolumn{2}{c}%
{\makebox[0pt]{\sphinxtablecontinued{\tablename\ \thetable{} -- continued from previous page}}}\\
\hline

\endhead

\hline
\multicolumn{2}{r}{\makebox[0pt][r]{\sphinxtablecontinued{Continued on next page}}}\\
\endfoot

\endlastfoot

{\hyperref[\detokenize{StreamStats_DataPrep:StreamStats_DataPrep.AdjustAccum.getParameterInfo}]{\sphinxcrossref{\sphinxcode{\sphinxupquote{getParameterInfo}}}}}(self)
&
Adjust flow accumulation grid inputs.
\\
\hline
\end{longtable}\sphinxatlongtableend\end{savenotes}
\index{getParameterInfo() (StreamStats\_DataPrep.AdjustAccum method)@\spxentry{getParameterInfo()}\spxextra{StreamStats\_DataPrep.AdjustAccum method}}

\begin{fulllineitems}
\phantomsection\label{\detokenize{StreamStats_DataPrep:StreamStats_DataPrep.AdjustAccum.getParameterInfo}}\pysiglinewithargsret{\sphinxbfcode{\sphinxupquote{getParameterInfo}}}{\emph{self}}{}
Adjust flow accumulation grid inputs.
\begin{quote}\begin{description}
\item[{Parameters}] \leavevmode\begin{description}
\item[{\sphinxstylestrong{Downstream Accumulation Grid}}] \leavevmode{[}DERasterDataset{]}
Downstream raster dataset to correct.

\item[{\sphinxstylestrong{Downstream Flow Direction Grid}}] \leavevmode{[}DERasterDataset{]}
Downstream flow direction grid.

\item[{\sphinxstylestrong{Upstream Flow Accumulation Grids}}] \leavevmode{[}DERasterDataset{]}
Upstream flow accumulation grids to correct the downstream grid with.

\item[{\sphinxstylestrong{Upstream Flow Direction Grids}}] \leavevmode{[}DERasterDataset{]}
Upstream flow direction grids corresponding to the grids listed above.

\item[{\sphinxstylestrong{Workspace}}] \leavevmode{[}Workspace (Geodatabase){]}
Geodatabase to work in.

\end{description}

\item[{Returns}] \leavevmode\begin{description}
\item[{\sphinxstylestrong{parameters}}] \leavevmode{[}list{]}
List of input parameters passed to the execute method.

\end{description}

\end{description}\end{quote}

\end{fulllineitems}


\end{fulllineitems}



\section{Adjust Accumulation Simple}
\label{\detokenize{StreamStats_DataPrep:adjust-accumulation-simple}}\index{AdjustAccumSimp (class in StreamStats\_DataPrep)@\spxentry{AdjustAccumSimp}\spxextra{class in StreamStats\_DataPrep}}

\begin{fulllineitems}
\phantomsection\label{\detokenize{StreamStats_DataPrep:StreamStats_DataPrep.AdjustAccumSimp}}\pysigline{\sphinxbfcode{\sphinxupquote{class }}\sphinxcode{\sphinxupquote{StreamStats\_DataPrep.}}\sphinxbfcode{\sphinxupquote{AdjustAccumSimp}}}
Simply adjust a flow accumulation grid.

This tool is a wrapper on {\hyperref[\detokenize{make_hydrodem:make_hydrodem.adjust_accum_simple}]{\sphinxcrossref{\sphinxcode{\sphinxupquote{make\_hydrodem.adjust\_accum\_simple()}}}}}.
\subsubsection*{Methods}


\begin{savenotes}\sphinxatlongtablestart\begin{longtable}{\X{1}{2}\X{1}{2}}
\hline

\endfirsthead

\multicolumn{2}{c}%
{\makebox[0pt]{\sphinxtablecontinued{\tablename\ \thetable{} -- continued from previous page}}}\\
\hline

\endhead

\hline
\multicolumn{2}{r}{\makebox[0pt][r]{\sphinxtablecontinued{Continued on next page}}}\\
\endfoot

\endlastfoot

{\hyperref[\detokenize{StreamStats_DataPrep:StreamStats_DataPrep.AdjustAccumSimp.getParameterInfo}]{\sphinxcrossref{\sphinxcode{\sphinxupquote{getParameterInfo}}}}}(self)
&
Simple flow accumulation grid adjustment inputs.
\\
\hline
\end{longtable}\sphinxatlongtableend\end{savenotes}
\index{getParameterInfo() (StreamStats\_DataPrep.AdjustAccumSimp method)@\spxentry{getParameterInfo()}\spxextra{StreamStats\_DataPrep.AdjustAccumSimp method}}

\begin{fulllineitems}
\phantomsection\label{\detokenize{StreamStats_DataPrep:StreamStats_DataPrep.AdjustAccumSimp.getParameterInfo}}\pysiglinewithargsret{\sphinxbfcode{\sphinxupquote{getParameterInfo}}}{\emph{self}}{}
Simple flow accumulation grid adjustment inputs.
\begin{quote}\begin{description}
\item[{Parameters}] \leavevmode\begin{description}
\item[{\sphinxstylestrong{Inlet Point}}] \leavevmode{[}GPFeatureLayer{]}
Point feature class indicating the inlet to the downstream hydrologic unit.

\item[{\sphinxstylestrong{Flow Direction Grid}}] \leavevmode{[}DERasterBand{]}
Flow direction grid of the downstream hydrologic unit.

\item[{\sphinxstylestrong{Flow Accumulation Grid}}] \leavevmode{[}DERasterBand{]}
Flow accumuation grid of the downstream hydrologic unit.

\item[{\sphinxstylestrong{HydroDEM}}] \leavevmode{[}DERasterBand{]}
Downstream hydrologic unit hydro-enforced digital elevation model.

\item[{\sphinxstylestrong{Output FAC}}] \leavevmode{[}DERasterBand{]}
Corrected flow accumuation grid, defaults to hydrodemfac\_global.

\item[{\sphinxstylestrong{Adjustment Value}}] \leavevmode{[}GPString{]}
Upstream flow accumulation value to correct the downstream hydrologic unit with, defaults to 150000 grid cells.

\end{description}

\item[{Returns}] \leavevmode\begin{description}
\item[{\sphinxstylestrong{parameters}}] \leavevmode{[}list{]}
List of input parameters passed to the execute method.

\end{description}

\end{description}\end{quote}

\end{fulllineitems}


\end{fulllineitems}



\section{Post Hydrodem}
\label{\detokenize{StreamStats_DataPrep:post-hydrodem}}\index{posthydrodem (class in StreamStats\_DataPrep)@\spxentry{posthydrodem}\spxextra{class in StreamStats\_DataPrep}}

\begin{fulllineitems}
\phantomsection\label{\detokenize{StreamStats_DataPrep:StreamStats_DataPrep.posthydrodem}}\pysigline{\sphinxbfcode{\sphinxupquote{class }}\sphinxcode{\sphinxupquote{StreamStats\_DataPrep.}}\sphinxbfcode{\sphinxupquote{posthydrodem}}}
ArcHydro processing using the hydro-enforced digital elevation model and resultant flow direction and flow accumulation grids.

This tool is a wrapper on {\hyperref[\detokenize{make_hydrodem:make_hydrodem.postHydroDEM}]{\sphinxcrossref{\sphinxcode{\sphinxupquote{make\_hydrodem.postHydroDEM()}}}}}.
\subsubsection*{Notes}

This tool only functions with ArcMap / Python 2, ArcPro / Python 3 are currently not supported.
\subsubsection*{Methods}


\begin{savenotes}\sphinxatlongtablestart\begin{longtable}{\X{1}{2}\X{1}{2}}
\hline

\endfirsthead

\multicolumn{2}{c}%
{\makebox[0pt]{\sphinxtablecontinued{\tablename\ \thetable{} -- continued from previous page}}}\\
\hline

\endhead

\hline
\multicolumn{2}{r}{\makebox[0pt][r]{\sphinxtablecontinued{Continued on next page}}}\\
\endfoot

\endlastfoot

{\hyperref[\detokenize{StreamStats_DataPrep:StreamStats_DataPrep.posthydrodem.getParameterInfo}]{\sphinxcrossref{\sphinxcode{\sphinxupquote{getParameterInfo}}}}}(self)
&
Post hydro-enforcement processing inputs.
\\
\hline
\end{longtable}\sphinxatlongtableend\end{savenotes}
\index{getParameterInfo() (StreamStats\_DataPrep.posthydrodem method)@\spxentry{getParameterInfo()}\spxextra{StreamStats\_DataPrep.posthydrodem method}}

\begin{fulllineitems}
\phantomsection\label{\detokenize{StreamStats_DataPrep:StreamStats_DataPrep.posthydrodem.getParameterInfo}}\pysiglinewithargsret{\sphinxbfcode{\sphinxupquote{getParameterInfo}}}{\emph{self}}{}
Post hydro-enforcement processing inputs.
\begin{quote}\begin{description}
\item[{Parameters}] \leavevmode\begin{description}
\item[{\sphinxstylestrong{Workspace}}] \leavevmode{[}DEWorkspace (Geodatabase){]}
Geodatabase to work in.

\item[{\sphinxstylestrong{hydrodemfac}}] \leavevmode{[}DERasterDataset{]}
Hydro-enforced flow accumulation grid.

\item[{\sphinxstylestrong{hydrodemfdr}}] \leavevmode{[}DERasterDataset{]}
Hydro-enforced flow direction grid.

\item[{\sphinxstylestrong{str threshold}}] \leavevmode{[}GPLong{]}
Stream threshold in raster cells.

\item[{\sphinxstylestrong{str900 threshold}}] \leavevmode{[}GPLong{]}
str900 grid threshold, in raster cells.

\item[{\sphinxstylestrong{Sink Link}}] \leavevmode{[}DERasterBand{]}
Sink link raster name.

\end{description}

\item[{Returns}] \leavevmode\begin{description}
\item[{\sphinxstylestrong{parameters}}] \leavevmode{[}list{]}
List of input parameters passed to the execute method.

\end{description}

\end{description}\end{quote}

\end{fulllineitems}


\end{fulllineitems}



\chapter{StreamStats DataPrep Library}
\label{\detokenize{modules:streamstats-dataprep-library}}\label{\detokenize{modules::doc}}

\section{databaseSetup Module}
\label{\detokenize{databaseSetup:module-databaseSetup}}\label{\detokenize{databaseSetup:databasesetup-module}}\label{\detokenize{databaseSetup::doc}}\index{databaseSetup (module)@\spxentry{databaseSetup}\spxextra{module}}
This library creates the folder structure and does not data management to facilitate preparing data for use in StreamStats.
\index{databaseSetup() (in module databaseSetup)@\spxentry{databaseSetup()}\spxextra{in module databaseSetup}}

\begin{fulllineitems}
\phantomsection\label{\detokenize{databaseSetup:databaseSetup.databaseSetup}}\pysiglinewithargsret{\sphinxcode{\sphinxupquote{databaseSetup.}}\sphinxbfcode{\sphinxupquote{databaseSetup}}}{\emph{output\_workspace}, \emph{output\_gdb\_name}, \emph{hu\_dataset}, \emph{hu8\_field}, \emph{hu12\_field}, \emph{hucbuffer}, \emph{nhd\_path}, \emph{elevation\_projection\_template}, \emph{alt\_buff}, \emph{version=None}}{}
Set up the local folders and copy hydrography data.

This tool creates folder cooresponding to each local hydrologic unit and fills those folders with the flowlines, inwalls, and outwalls that will be used later to hydro-enforce the digital elevation model for each hydrologic unit. This tool also creates a global geodatabase with a feature class for the whole domain.
\begin{quote}\begin{description}
\item[{Parameters}] \leavevmode\begin{description}
\item[{\sphinxstylestrong{output\_workspace}}] \leavevmode{[}str{]}
Output directory for processing to occure in.

\item[{\sphinxstylestrong{output\_gdb\_name}}] \leavevmode{[}str{]}
Global file geodatabase to be created.

\item[{\sphinxstylestrong{hu\_dataset}}] \leavevmode{[}str{]}
Feature class that defines local folder geographic boundaries.

\item[{\sphinxstylestrong{hu8\_field}}] \leavevmode{[}str{]}
Field name in “hu\_dataset” to dissolve boundaries to local folder extents.

\item[{\sphinxstylestrong{hu12\_field}}] \leavevmode{[}str{]}
Field name in “hu\_dataset” to generate inwalls from.

\item[{\sphinxstylestrong{hucbuffer}}] \leavevmode{[}str{]}
Distance to buffer local folder bounds in map units.

\item[{\sphinxstylestrong{nhd\_path}}] \leavevmode{[}str{]}
Path to workspace containing NHD geodatabases.

\item[{\sphinxstylestrong{elevation\_projection\_template}}] \leavevmode{[}str{]}
Path to DEM file to use as a projection template.

\item[{\sphinxstylestrong{alt\_buff}}] \leavevmode{[}str{]}
Alternative buffer to use on local folder boundaries.

\item[{\sphinxstylestrong{version}}] \leavevmode{[}str{]}
Package version number.

\end{description}

\item[{Returns}] \leavevmode\begin{description}
\item[{None}] \leavevmode
\end{description}

\end{description}\end{quote}
\subsubsection*{Notes}

As this tool moves through each local hydrologic unit it searches the \sphinxstyleemphasis{nhd\_path} for a geodatabase with hydrography data with the same HUC-4 as the local hydrologic unit. If this cannot be found the tool will skip that local hydrologic unit. Non-NHD hydrography data can be used with this tool, but it must be named and organized in the same way that NHD hydrography is.

\end{fulllineitems}



\section{elevationTools Module}
\label{\detokenize{elevationTools:module-elevationTools}}\label{\detokenize{elevationTools:elevationtools-module}}\label{\detokenize{elevationTools::doc}}\index{elevationTools (module)@\spxentry{elevationTools}\spxextra{module}}
This library is a collection of functions to create an seamless mosaic of digital elevation models for an area of interest, extracts a hydrologic unit (local folder) from this mosaic, checks for and fills no data values in the digital elevation model, projects the digital elevation model to the target projection, and scales the elevation values for better data storage.
\index{checkNoData() (in module elevationTools)@\spxentry{checkNoData()}\spxextra{in module elevationTools}}

\begin{fulllineitems}
\phantomsection\label{\detokenize{elevationTools:elevationTools.checkNoData}}\pysiglinewithargsret{\sphinxcode{\sphinxupquote{elevationTools.}}\sphinxbfcode{\sphinxupquote{checkNoData}}}{\emph{InGrid}, \emph{tmpLoc}, \emph{OutPolys\_shp}, \emph{version=None}}{}
Generates a feature class of no data values.
\begin{quote}\begin{description}
\item[{Parameters}] \leavevmode\begin{description}
\item[{\sphinxstylestrong{InGrid}}] \leavevmode{[}Raster{]}
Input DEM grid to search for no data values

\item[{\sphinxstylestrong{tmpLoc}}] \leavevmode{[}str{]}
Path to workspace

\item[{\sphinxstylestrong{OutPoly\_shp}}] \leavevmode{[}str{]}
Name for output feature class

\item[{\sphinxstylestrong{version}}] \leavevmode{[}str, optional{]}
StreamStats DataPrepTools version

\end{description}

\item[{Returns}] \leavevmode\begin{description}
\item[{\sphinxstylestrong{featCount}}] \leavevmode{[}int{]}
Count of no data features generated.

\end{description}

\end{description}\end{quote}

\end{fulllineitems}

\index{compareSpatialRefUnits() (in module elevationTools)@\spxentry{compareSpatialRefUnits()}\spxextra{in module elevationTools}}

\begin{fulllineitems}
\phantomsection\label{\detokenize{elevationTools:elevationTools.compareSpatialRefUnits}}\pysiglinewithargsret{\sphinxcode{\sphinxupquote{elevationTools.}}\sphinxbfcode{\sphinxupquote{compareSpatialRefUnits}}}{\emph{grd}}{}
Compare horizontal and vertical units from a raster dataset. Returns True if units are the same, returns False if they are different.
\begin{quote}\begin{description}
\item[{Parameters}] \leavevmode\begin{description}
\item[{\sphinxstylestrong{grd}}] \leavevmode{[}str{]}
Path to raster dataset.

\end{description}

\item[{Returns}] \leavevmode\begin{description}
\item[{\sphinxstylestrong{sameUnits}}] \leavevmode{[}bool{]}
True if units are the same, False if not.

\end{description}

\end{description}\end{quote}

\end{fulllineitems}

\index{elevIndex() (in module elevationTools)@\spxentry{elevIndex()}\spxextra{in module elevationTools}}

\begin{fulllineitems}
\phantomsection\label{\detokenize{elevationTools:elevationTools.elevIndex}}\pysiglinewithargsret{\sphinxcode{\sphinxupquote{elevationTools.}}\sphinxbfcode{\sphinxupquote{elevIndex}}}{\emph{OutLoc}, \emph{rcName}, \emph{coordsysRaster}, \emph{InputELEVDATAws}, \emph{version=None}}{}
Make a raster mosaic dataset of the digital elevation models found in \sphinxstyleemphasis{InputELEVDATAws}.
\begin{quote}\begin{description}
\item[{Parameters}] \leavevmode\begin{description}
\item[{\sphinxstylestrong{OutLoc}}] \leavevmode{[}str{]}
Path to output location for the raster catalog.

\item[{\sphinxstylestrong{rcName}}] \leavevmode{[}str{]}
Name of the output mosaic raster dataset.

\item[{\sphinxstylestrong{coordsysRaster}}] \leavevmode{[}str{]}
Path to raster from which to base the mosaic dataset’s coordinate system.

\item[{\sphinxstylestrong{InputELEVDATAws}}] \leavevmode{[}str{]}
Path to workspace containing the elevation data to be included in the mosaic raster dataset. Rasters in this workspace should be either geoTiffs or ESRI grids. Rasters must be unzipped, but they can be in subfolders.

\item[{\sphinxstylestrong{version}}] \leavevmode{[}str, optional{]}
StreamStats DataPrepTools version number.

\end{description}

\item[{Returns}] \leavevmode\begin{description}
\item[{None}] \leavevmode
\end{description}

\end{description}\end{quote}

\end{fulllineitems}

\index{extractPoly() (in module elevationTools)@\spxentry{extractPoly()}\spxextra{in module elevationTools}}

\begin{fulllineitems}
\phantomsection\label{\detokenize{elevationTools:elevationTools.extractPoly}}\pysiglinewithargsret{\sphinxcode{\sphinxupquote{elevationTools.}}\sphinxbfcode{\sphinxupquote{extractPoly}}}{\emph{Input\_Workspace}, \emph{nedindx}, \emph{clpfeat}, \emph{OutGrd}, \emph{version=None}}{}
Extracts watershed DEM from a raster catalogue of tiles based on a watershed feature.
\begin{quote}\begin{description}
\item[{Parameters}] \leavevmode\begin{description}
\item[{\sphinxstylestrong{Input\_Workspace}}] \leavevmode{[}str{]}
Path to the workspace to work in.

\item[{\sphinxstylestrong{nedindx}}] \leavevmode{[}str{]}
Path to the elevation data mosaic dataset.

\item[{\sphinxstylestrong{clpfeat}}] \leavevmode{[}str{]}
Path to the clipping feature.

\item[{\sphinxstylestrong{OutGrd}}] \leavevmode{[}str{]}
Name of the output grid to be generated in Input\_Workspace.

\item[{\sphinxstylestrong{version}}] \leavevmode{[}str, optional{]}
StreamStats DataPrepTools version number.

\end{description}

\item[{Returns}] \leavevmode\begin{description}
\item[{None}] \leavevmode
\end{description}

\end{description}\end{quote}

\end{fulllineitems}

\index{fillNoData() (in module elevationTools)@\spxentry{fillNoData()}\spxextra{in module elevationTools}}

\begin{fulllineitems}
\phantomsection\label{\detokenize{elevationTools:elevationTools.fillNoData}}\pysiglinewithargsret{\sphinxcode{\sphinxupquote{elevationTools.}}\sphinxbfcode{\sphinxupquote{fillNoData}}}{\emph{workspace}, \emph{InGrid}, \emph{OutGrid}, \emph{version=None}}{}
Replaces NODATA values in a grid with mean values within 3x3 window.
\begin{quote}\begin{description}
\item[{Parameters}] \leavevmode\begin{description}
\item[{\sphinxstylestrong{workspace}}] \leavevmode{[}str{]}
Path to the workspace to work in.

\item[{\sphinxstylestrong{InGrid}}] \leavevmode{[}str{]}
Name of the input grid to be filled.

\item[{\sphinxstylestrong{OutGrid}}] \leavevmode{[}str{]}
Name of the output grid.

\item[{\sphinxstylestrong{Version}}] \leavevmode{[}str, optional{]}
Code version

\end{description}

\item[{Returns}] \leavevmode\begin{description}
\item[{None}] \leavevmode
\end{description}

\end{description}\end{quote}
\subsubsection*{Notes}

May be run repeatedly to fill in areas wider than 2 cells. Note the output is floating point, even if the input is integer. Note this will expand the data area of the grid around the outer edges of data, in addition to filling in NODATA gaps in the interior of the grid.

Converted from model builder to arcpy, Theodore Barnhart, \sphinxhref{mailto:tbarnhart@usgs.gov}{tbarnhart@usgs.gov}, 20190222

\end{fulllineitems}

\index{projScale() (in module elevationTools)@\spxentry{projScale()}\spxextra{in module elevationTools}}

\begin{fulllineitems}
\phantomsection\label{\detokenize{elevationTools:elevationTools.projScale}}\pysiglinewithargsret{\sphinxcode{\sphinxupquote{elevationTools.}}\sphinxbfcode{\sphinxupquote{projScale}}}{\emph{Input\_Workspace}, \emph{InGrd}, \emph{OutGrd}, \emph{OutCoordsys}, \emph{OutCellSize}, \emph{RegistrationPoint}, \emph{scaleFact=100}, \emph{version=None}}{}~\begin{description}
\item[{Projects a NED grid to a user-specified coordinate system, handling cell registration. Converts}] \leavevmode
output grid to centimeters (multiplies by 100 and rounds).

\end{description}
\begin{quote}\begin{description}
\item[{Parameters}] \leavevmode\begin{description}
\item[{\sphinxstylestrong{Input\_Workspace}}] \leavevmode{[}str{]}
Path to input workspace.

\item[{\sphinxstylestrong{InGrd}}] \leavevmode{[}str{]}
Name of the grid to be projected and scaled.

\item[{\sphinxstylestrong{OutGrd}}] \leavevmode{[}str{]}
Name of the output grid.

\item[{\sphinxstylestrong{OutCoordsys}}] \leavevmode{[}str{]}
Path to the dataset to base the projection off of.

\item[{\sphinxstylestrong{OutCellSize}}] \leavevmode{[}int or float{]}
Cell size for output grid.

\item[{\sphinxstylestrong{RegistrationPoint}}] \leavevmode{[}str{]}
Registration point for output grid so all grids snap correctly. In the format “0 0” where the zeros are the x and y of the registration point.

\item[{\sphinxstylestrong{version}}] \leavevmode{[}str{]}
Stream Stats version number.

\end{description}

\item[{Returns}] \leavevmode\begin{description}
\item[{None}] \leavevmode
\end{description}

\end{description}\end{quote}

\end{fulllineitems}



\section{make\_hydrodem Module}
\label{\detokenize{make_hydrodem:module-make_hydrodem}}\label{\detokenize{make_hydrodem:make-hydrodem-module}}\label{\detokenize{make_hydrodem::doc}}\index{make\_hydrodem (module)@\spxentry{make\_hydrodem}\spxextra{module}}
Code to replicate the hydroDEM\_work\_mod.aml, agree.aml, and fill.aml scripts

Theodore Barnhart, \sphinxhref{mailto:tbarnhart@usgs.gov}{tbarnhart@usgs.gov}, 20190225

Reference: agree.aml
\index{SnapExtent() (in module make\_hydrodem)@\spxentry{SnapExtent()}\spxextra{in module make\_hydrodem}}

\begin{fulllineitems}
\phantomsection\label{\detokenize{make_hydrodem:make_hydrodem.SnapExtent}}\pysiglinewithargsret{\sphinxcode{\sphinxupquote{make\_hydrodem.}}\sphinxbfcode{\sphinxupquote{SnapExtent}}}{\emph{lExtent}, \emph{lRaster}}{}
Returns a given extent snapped to the passed raster.
\begin{quote}\begin{description}
\item[{Parameters}] \leavevmode\begin{description}
\item[{\sphinxstylestrong{lExtent}}] \leavevmode{[}str{]}
ESRI Arcpy extent string

\item[{\sphinxstylestrong{lRaster}}] \leavevmode{[}str{]}
Path to raster dataset

\end{description}

\item[{Returns}] \leavevmode\begin{description}
\item[{\sphinxstylestrong{extent}}] \leavevmode{[}str{]}
ESRI ArcPy extent string

\end{description}

\end{description}\end{quote}

\end{fulllineitems}

\index{adjust\_accum() (in module make\_hydrodem)@\spxentry{adjust\_accum()}\spxextra{in module make\_hydrodem}}

\begin{fulllineitems}
\phantomsection\label{\detokenize{make_hydrodem:make_hydrodem.adjust_accum}}\pysiglinewithargsret{\sphinxcode{\sphinxupquote{make\_hydrodem.}}\sphinxbfcode{\sphinxupquote{adjust\_accum}}}{\emph{facPth}, \emph{fdrPth}, \emph{upstreamFACpths}, \emph{upstreamFDRpths}, \emph{workspace}, \emph{version=None}}{}~\begin{quote}\begin{description}
\item[{Parameters}] \leavevmode\begin{description}
\item[{\sphinxstylestrong{facPth}}] \leavevmode{[}str{]}
Path to downstream flow accumulation grid

\item[{\sphinxstylestrong{fdrPth}}] \leavevmode{[}str{]}
Path to downstream flow direction grid

\item[{\sphinxstylestrong{upstreamFACpths}}] \leavevmode{[}list{]}
List of paths to upstream flow accumulation grids

\item[{\sphinxstylestrong{upstreamFDRpths}}] \leavevmode{[}list{]}
List of paths to upstream flow direction grids

\item[{\sphinxstylestrong{workspace}}] \leavevmode{[}str{]}
local geodatabase to work in.

\item[{\sphinxstylestrong{version}}] \leavevmode{[}str (optional){]}
Stream Stats datapreptool version number.

\end{description}

\end{description}\end{quote}

\end{fulllineitems}

\index{adjust\_accum\_simple() (in module make\_hydrodem)@\spxentry{adjust\_accum\_simple()}\spxextra{in module make\_hydrodem}}

\begin{fulllineitems}
\phantomsection\label{\detokenize{make_hydrodem:make_hydrodem.adjust_accum_simple}}\pysiglinewithargsret{\sphinxcode{\sphinxupquote{make\_hydrodem.}}\sphinxbfcode{\sphinxupquote{adjust\_accum\_simple}}}{\emph{ptin}, \emph{fdrin}, \emph{facin}, \emph{filin}, \emph{facout}, \emph{incrval}, \emph{version=None}}{}
Simple drainage adjust method.

Original coding by Al Rea (2010) \sphinxhref{mailto:ahrea@usgs.gov}{ahrea@usgs.gov}
Updated to arcPy by Theodore Barnhart (2019) \sphinxhref{mailto:tbarnhart@usgs.gov}{tbarnhart@usgs.gov}

Adds a value to the flow accumulation grid given an input point using a least-cost-path to coascalde down through the flow direction grid.
\begin{quote}\begin{description}
\item[{Parameters}] \leavevmode\begin{description}
\item[{\sphinxstylestrong{ptin}}] \leavevmode{[}str (feature class){]}
Point feature class representing one inlet to the downstream DEM.

\item[{\sphinxstylestrong{fdrin}}] \leavevmode{[}str (raster){]}
Flow direction raster

\item[{\sphinxstylestrong{facin}}] \leavevmode{[}str (raster){]}
Name of the flow accumulation raster

\item[{\sphinxstylestrong{filin}}] \leavevmode{[}str (raster){]}
Burned DEM to use as cost surface.

\item[{\sphinxstylestrong{facout}}] \leavevmode{[}str (raster){]}
Output name of adjusted FAC grid.

\item[{\sphinxstylestrong{incrval}}] \leavevmode{[}int{]}
Value to adjust the downstream FAC grid by.

\item[{\sphinxstylestrong{version}}] \leavevmode{[}str{]}
Stream Stats version number

\end{description}

\item[{Returns}] \leavevmode\begin{description}
\item[{None}] \leavevmode
\end{description}

\end{description}\end{quote}

\end{fulllineitems}

\index{agree() (in module make\_hydrodem)@\spxentry{agree()}\spxextra{in module make\_hydrodem}}

\begin{fulllineitems}
\phantomsection\label{\detokenize{make_hydrodem:make_hydrodem.agree}}\pysiglinewithargsret{\sphinxcode{\sphinxupquote{make\_hydrodem.}}\sphinxbfcode{\sphinxupquote{agree}}}{\emph{origdem}, \emph{dendrite}, \emph{agreebuf}, \emph{agreesmooth}, \emph{agreesharp}}{}
Agree function from AGREE.aml

Original function by Ferdi Hellweger, \sphinxurl{http://www.ce.utexas.edu/prof/maidment/gishydro/ferdi/research/agree/agree.html}

recoded by Theodore Barnhart, \sphinxhref{mailto:tbarnhart@usgs.gov}{tbarnhart@usgs.gov}, 20190225

— Creation Information —

Name: agree.aml
Version: 1.1
Date: 10/13/96
Author: Ferdi Hellweger
\begin{quote}

Center for Research in Water Resources
The University of Texas at Austin
\sphinxhref{mailto:ferdi@crwr.utexas.edu}{ferdi@crwr.utexas.edu}
\end{quote}

— Purpose/Description —

AGREE is a surface reconditioning system for Digital Elevation Models (DEMs).
The system adjusts the surface elevation of the DEM to be consistent with a
vector coverage.  The vecor coverage can be a stream or ridge line coverage.
\begin{quote}\begin{description}
\item[{Parameters}] \leavevmode\begin{description}
\item[{\sphinxstylestrong{origdem}}] \leavevmode{[}arcpy.sa Raster{]}
Original DEM with the desired cell size, oelevgrid in original script

\item[{\sphinxstylestrong{dendrite}}] \leavevmode{[}arcpy.sa Raster{]}
Dendrite feature layer to adjust the DEM, vectcov in the original script

\item[{\sphinxstylestrong{agreebuf}}] \leavevmode{[}float{]}
Buffer smoothing distance (same units as the horizontal), buffer in original script

\item[{\sphinxstylestrong{agreesmooth}}] \leavevmode{[}float{]}
Smoothing distance (same units as the vertical), smoothdist in the original script

\item[{\sphinxstylestrong{agreesharp}}] \leavevmode{[}float{]}
Distance for sharp feature (same units as the vertical), sharpdist in the original script

\end{description}

\item[{Returns}] \leavevmode\begin{description}
\item[{\sphinxstylestrong{elevgrid}}] \leavevmode{[}arcpy.sa Raster{]}
conditioned elevation grid returned as a arcpy.sa Raster object

\end{description}

\end{description}\end{quote}

\end{fulllineitems}

\index{bathymetricGradient() (in module make\_hydrodem)@\spxentry{bathymetricGradient()}\spxextra{in module make\_hydrodem}}

\begin{fulllineitems}
\phantomsection\label{\detokenize{make_hydrodem:make_hydrodem.bathymetricGradient}}\pysiglinewithargsret{\sphinxcode{\sphinxupquote{make\_hydrodem.}}\sphinxbfcode{\sphinxupquote{bathymetricGradient}}}{\emph{workspace}, \emph{snapGrid}, \emph{hucPoly}, \emph{hydrographyArea}, \emph{hydrographyFlowline}, \emph{hydrographyWaterbody}, \emph{cellsize}, \emph{version=None}}{}
Generates the input datasets for enforcing a bathymetic gradient in hydroDEM (bowling).
\begin{description}
\item[{Originally:}] \leavevmode
ssbowling.py
Created on: Wed Jan 31 2007 01:16:48 PM
Author:  Martyn Smith
USGS New York Water Science Center Troy, NY

\end{description}

Updated to Arcpy, 20190222, Theodore Barnhart, \sphinxhref{mailto:tbarnhart@usgs.gov}{tbarnhart@usgs.gov}

This script takes a set of NHD Hydrography Datasets, extracts the appropriate
features and converts them to rasters for the Bathymetric Gradient (bowling) inputs to HydroDEM
\begin{quote}\begin{description}
\item[{Parameters}] \leavevmode\begin{description}
\item[{\sphinxstylestrong{workspace}}] \leavevmode{[}str{]}
\item[{\sphinxstylestrong{snapGrid}}] \leavevmode{[}str{]}
\item[{\sphinxstylestrong{hucPoly}}] \leavevmode{[}str{]}
\item[{\sphinxstylestrong{hydrographyArea}}] \leavevmode{[}str{]}
\item[{\sphinxstylestrong{hydrographyFlowline}}] \leavevmode{[}str{]}
\item[{\sphinxstylestrong{hydrographyWaterbody}}] \leavevmode{[}str{]}
\item[{\sphinxstylestrong{cellsize}}] \leavevmode{[}str{]}
\item[{\sphinxstylestrong{version}}] \leavevmode{[}str{]}
Package version number

\end{description}

\end{description}\end{quote}

\end{fulllineitems}

\index{coastaldem() (in module make\_hydrodem)@\spxentry{coastaldem()}\spxextra{in module make\_hydrodem}}

\begin{fulllineitems}
\phantomsection\label{\detokenize{make_hydrodem:make_hydrodem.coastaldem}}\pysiglinewithargsret{\sphinxcode{\sphinxupquote{make\_hydrodem.}}\sphinxbfcode{\sphinxupquote{coastaldem}}}{\emph{Input\_Workspace}, \emph{grdNamePth}, \emph{InFeatureClass}, \emph{OutRaster}, \emph{seaLevel}}{}
Sets elevations for water and other areas in DEM
\begin{description}
\item[{Originally:}] \leavevmode
Al Rea, \sphinxhref{mailto:ahrea@usgs.gov}{ahrea@usgs.gov}, 05/01/2010, original coding
ahrea, 10/30/2010 updated with more detailed comments
Theodore Barnhart, 20190225, \sphinxhref{mailto:tbarnhart@usgs.gov}{tbarnhart@usgs.gov}, updated to arcpy

\end{description}
\begin{quote}\begin{description}
\item[{Parameters}] \leavevmode\begin{description}
\item[{\sphinxstylestrong{Input\_Workspace}}] \leavevmode{[}str{]}
Input workspace, output raster will be written here.

\item[{\sphinxstylestrong{grdNamePth}}] \leavevmode{[}str{]}
Input DEM grid.

\item[{\sphinxstylestrong{InFeatureClass}}] \leavevmode{[}str{]}
LandSea feature class.

\item[{\sphinxstylestrong{OutRaster}}] \leavevmode{[}str{]}
Output DEM grid name.

\item[{\sphinxstylestrong{seaLevel}}] \leavevmode{[}float{]}
Elevation at which to make the sea

\end{description}

\item[{Returns}] \leavevmode\begin{description}
\item[{\sphinxstylestrong{OutRaster}}] \leavevmode{[}raster{]}
Output raster written to the workspace.

\end{description}

\end{description}\end{quote}

\end{fulllineitems}

\index{hydrodem() (in module make\_hydrodem)@\spxentry{hydrodem()}\spxextra{in module make\_hydrodem}}

\begin{fulllineitems}
\phantomsection\label{\detokenize{make_hydrodem:make_hydrodem.hydrodem}}\pysiglinewithargsret{\sphinxcode{\sphinxupquote{make\_hydrodem.}}\sphinxbfcode{\sphinxupquote{hydrodem}}}{\emph{outdir}, \emph{huc8cov}, \emph{origdemPth}, \emph{dendrite}, \emph{snap\_grid}, \emph{bowl\_polys}, \emph{bowl\_lines}, \emph{inwall}, \emph{drainplug}, \emph{buffdist}, \emph{inwallbuffdist}, \emph{inwallht}, \emph{outwallht}, \emph{agreebuf}, \emph{agreesmooth}, \emph{agreesharp}, \emph{bowldepth}, \emph{cellsz}, \emph{scratchWorkspace}, \emph{version=None}}{}
Hydro-enforce a DEM

This aml is used by the National StreamStats Team as the optimal
approach for preparing a state’s physiographic datasets for watershed delineations.
It takes as input, a 10-meter (or 30-foot) DEM, and enforces this data to recognize
NHD hydrography as truth.  WBD can also be recognized as truth if avaialable for a
given state/region. This aml assumes that the DEM has first been projected to a
state’s projection of choice. This aml prepares data to be used in the Archydro
data model (the GIS database environment for National StreamStats).

The specified \textless{}8-digit HUC\textgreater{} should have a path associated with it in the variable 
settings section near the top of this aml.  The value entered will create a workspace
with this HUC id as it’s name, and copy all output datasets into the new workspace.
If the workspace already Exists, it should be empty before running this aml.

The snap\_grid is used to orient the origin coordinate of the output grids to align 
with neighboring HUC grids that have already been processed.  
Typically, this value is rounded to the nearest value
divisible by 10 (in cases where datsets are in units meters) or 30 (in cases where
datasets are in units feet).  The snap grid could be your input dem, if that grid
has already been rounded out (if topogrid was used and steps were followed on the 
nhd web page referenced above, then the input dem could be used).
\begin{quote}\begin{description}
\item[{Parameters}] \leavevmode\begin{description}
\item[{\sphinxstylestrong{outdir}}] \leavevmode{[}DEworkspace{]}
Working directory

\item[{\sphinxstylestrong{huc8cov}}] \leavevmode{[}DEFeatureClass{]}
Local division feature class, often HUC8, this will be the outer wall of the hydroDEM.

\item[{\sphinxstylestrong{origdemPth}}] \leavevmode{[}str{]}
Path to the orignial, projected DEM.

\item[{\sphinxstylestrong{dendrite}}] \leavevmode{[}str{]}
Path to the dendrite feature class to be used.

\item[{\sphinxstylestrong{snap\_grid}}] \leavevmode{[}str{]}
Path to a raster dataset to use as a snap\_grid to align all the watersheds, often the same as the DEM.

\item[{\sphinxstylestrong{bowl\_polys}}] \leavevmode{[}str{]}
Path to the bowling area raster generated from the bathymetric gradient tool.

\item[{\sphinxstylestrong{bowl\_lines}}] \leavevmode{[}str{]}
Path to the bowling line raster generated from the bathymetric gradient tool.

\item[{\sphinxstylestrong{inwall}}] \leavevmode{[}str{]}
Path to the feature class to be used for inwalling

\item[{\sphinxstylestrong{drainplug :}}] \leavevmode
Path to the feature class used for inserting sinks into the dataset

\item[{\sphinxstylestrong{buffdist}}] \leavevmode{[}float{]}
Distance to buffer the outer wall, same units as the projection.

\item[{\sphinxstylestrong{inwallbuffdist :}}] \leavevmode
Distance to buffer the inner walls, same units as the projection.

\item[{\sphinxstylestrong{inwallht :}}] \leavevmode
Inwall height, same units as the projection.

\item[{\sphinxstylestrong{outwallht :}}] \leavevmode
Inwall height, same units as the projection.

\item[{\sphinxstylestrong{agreebuf :}}] \leavevmode
AGREE function buffer distance, same units as the projection.

\item[{\sphinxstylestrong{agreesmooth :}}] \leavevmode
AGREE function smoothing distance, same units as the projection.

\item[{\sphinxstylestrong{agreesharp :}}] \leavevmode
AGREE function sharp distance, same units as the projection.

\item[{\sphinxstylestrong{bowldepth :}}] \leavevmode
Bowling depth, same units as the projection.

\item[{\sphinxstylestrong{cellsz :}}] \leavevmode
Cell size, same units as the projection.

\item[{\sphinxstylestrong{scratchWorkspace}}] \leavevmode{[}str{]}
Path to scratch workspace

\item[{\sphinxstylestrong{version}}] \leavevmode{[}str{]}
Package version number

\item[{\sphinxstylestrong{Returns (saved to outDIR)}}] \leavevmode
\item[{\sphinxstylestrong{——-}}] \leavevmode
\item[{\sphinxstylestrong{filldem}}] \leavevmode{[}raster{]}
hydro-enforced DEM raster grid saved to outDir

\item[{\sphinxstylestrong{fdirg}}] \leavevmode{[}raster{]}
HydroDEM FDR raster grid saved to outDir

\item[{\sphinxstylestrong{faccg}}] \leavevmode{[}raster{]}
HydroDEM FAC raster grid saved to outDir

\item[{\sphinxstylestrong{sink\_path}}] \leavevmode{[}feature class{]}
Sink feature class saved to outDir

\end{description}

\end{description}\end{quote}

\end{fulllineitems}

\index{moveRasters() (in module make\_hydrodem)@\spxentry{moveRasters()}\spxextra{in module make\_hydrodem}}

\begin{fulllineitems}
\phantomsection\label{\detokenize{make_hydrodem:make_hydrodem.moveRasters}}\pysiglinewithargsret{\sphinxcode{\sphinxupquote{make\_hydrodem.}}\sphinxbfcode{\sphinxupquote{moveRasters}}}{\emph{source}, \emph{dest}, \emph{rasters}, \emph{fmt=None}}{}
Move raster out of a working geodatabase to a destination folder.

\end{fulllineitems}

\index{postHydroDEM() (in module make\_hydrodem)@\spxentry{postHydroDEM()}\spxextra{in module make\_hydrodem}}

\begin{fulllineitems}
\phantomsection\label{\detokenize{make_hydrodem:make_hydrodem.postHydroDEM}}\pysiglinewithargsret{\sphinxcode{\sphinxupquote{make\_hydrodem.}}\sphinxbfcode{\sphinxupquote{postHydroDEM}}}{\emph{workspace}, \emph{facPth}, \emph{fdrPth}, \emph{thresh1}, \emph{thresh2}, \emph{sinksPth=None}, \emph{version=None}}{}
generate stream reaches, adjoint catchments, and drainage points
\begin{quote}\begin{description}
\item[{Parameters}] \leavevmode\begin{description}
\item[{\sphinxstylestrong{workspace}}] \leavevmode{[}str{]}
database-type workspace to output rasters and feature classes.

\item[{\sphinxstylestrong{facPth}}] \leavevmode{[}str{]}
Path to the flow accumulation grid produced by hydroDEM.

\item[{\sphinxstylestrong{fdrPth}}] \leavevmode{[}str{]}
Path to the flow direction grid produced by hydroDEM.

\item[{\sphinxstylestrong{thresh1}}] \leavevmode{[}int{]}
Threshold used to produce the str grid, in raster cells, usually equal to 15,000,000 m\$\textasciicircum{}2\$.

\item[{\sphinxstylestrong{thresh2}}] \leavevmode{[}int{]}
Threshold used to produce the str900 grid, in raster cells, usually equal to 810,000 m\$\textasciicircum{}2\$.

\item[{\sphinxstylestrong{sinksPth}}] \leavevmode{[}str (optional){]}
Path to the snklnk grid, optional.

\item[{\sphinxstylestrong{version}}] \leavevmode{[}str (optional){]}
StreamStats DataPrepTools version to be printed.

\end{description}

\item[{Returns}] \leavevmode\begin{description}
\item[{None}] \leavevmode
\end{description}

\end{description}\end{quote}

\end{fulllineitems}



\section{topo\_grid Module}
\label{\detokenize{topo_grid:module-topo_grid}}\label{\detokenize{topo_grid:topo-grid-module}}\label{\detokenize{topo_grid::doc}}\index{topo\_grid (module)@\spxentry{topo\_grid}\spxextra{module}}\index{topogrid() (in module topo\_grid)@\spxentry{topogrid()}\spxextra{in module topo\_grid}}

\begin{fulllineitems}
\phantomsection\label{\detokenize{topo_grid:topo_grid.topogrid}}\pysiglinewithargsret{\sphinxcode{\sphinxupquote{topo\_grid.}}\sphinxbfcode{\sphinxupquote{topogrid}}}{\emph{workspace}, \emph{huc8}, \emph{buffdist}, \emph{dendrite}, \emph{dem}, \emph{cellSize}, \emph{vipPer}, \emph{snapgrid=None}, \emph{huc12=None}}{}~\begin{quote}\begin{description}
\item[{Parameters}] \leavevmode\begin{description}
\item[{\sphinxstylestrong{workspace}}] \leavevmode{[}str{]}
Path to geodatabase

\item[{\sphinxstylestrong{huc8}}] \leavevmode{[}str{]}
Path to huc8 feature class

\item[{\sphinxstylestrong{buffdist}}] \leavevmode{[}int{]}
Distance to buffer huc8 in horizontal map units

\item[{\sphinxstylestrong{dendrite}}] \leavevmode{[}str{]}
Path to flowline dendrite feature class

\item[{\sphinxstylestrong{dem}}] \leavevmode{[}str{]}
Path to buffered, scalled, and projected DEM

\item[{\sphinxstylestrong{cellSize}}] \leavevmode{[}int{]}
Output cell size

\item[{\sphinxstylestrong{vipPer}}] \leavevmode{[}int{]}
VIP thining value

\item[{\sphinxstylestrong{snapgrid}}] \leavevmode{[}str (optional){]}
Path to snapgrid to use instead of input DEM.

\item[{\sphinxstylestrong{huc12}}] \leavevmode{[}list{]}
List of paths to HUC12 values if the huc8 doesn’t work

\end{description}

\item[{Returns}] \leavevmode\begin{description}
\item[{None}] \leavevmode
\end{description}

\end{description}\end{quote}
\subsubsection*{Notes}

See \sphinxurl{https://support.esri.com/en/technical-article/000004588}

\end{fulllineitems}



\chapter{Indices and tables}
\label{\detokenize{index:indices-and-tables}}\begin{itemize}
\item {} 
\DUrole{xref,std,std-ref}{genindex}

\item {} 
\DUrole{xref,std,std-ref}{modindex}

\item {} 
\DUrole{xref,std,std-ref}{search}

\end{itemize}


\renewcommand{\indexname}{Python Module Index}
\begin{sphinxtheindex}
\let\bigletter\sphinxstyleindexlettergroup
\bigletter{d}
\item\relax\sphinxstyleindexentry{databaseSetup}\sphinxstyleindexpageref{databaseSetup:\detokenize{module-databaseSetup}}
\indexspace
\bigletter{e}
\item\relax\sphinxstyleindexentry{elevationTools}\sphinxstyleindexpageref{elevationTools:\detokenize{module-elevationTools}}
\indexspace
\bigletter{m}
\item\relax\sphinxstyleindexentry{make\_hydrodem}\sphinxstyleindexpageref{make_hydrodem:\detokenize{module-make_hydrodem}}
\indexspace
\bigletter{t}
\item\relax\sphinxstyleindexentry{topo\_grid}\sphinxstyleindexpageref{topo_grid:\detokenize{module-topo_grid}}
\end{sphinxtheindex}

\renewcommand{\indexname}{Index}
\printindex
\end{document}